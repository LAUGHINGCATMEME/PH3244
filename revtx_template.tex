\documentclass[%
sor,
 jor,
 amsmath,amssymb,
 reprint,
]{revtex4-2}

\usepackage{booktabs}
\usepackage{float}
\usepackage{caption}
\usepackage{amsmath, amssymb, amsfonts}
\usepackage{siunitx}
\usepackage{graphicx}
\usepackage{circuitikz}

\begin{document}
\title{PH3244\\Experiment - \\ NAME}
\author{Aumshree P. Shah\\20231059}
\altaffiliation{\color{red}aumshree.pinkalbenshah@students.iiserpune.ac.in}
\date{\today}
\begin{abstract}
\centering
ABSTREACXCT
\end{abstract}
\maketitle
\tableofcontents
\vspace*{\fill}
\pagebreak
\section{Theory and Procedure}
\subsection{Theory}
theory
\subsection{Procedure}
procedure	

\section{Observation}

\subsection{obs1}
\subsection{obs2}
\subsection{Comments}

\section{Calculations and Characteristics}
\subsection{Sources of Error}
\subsubsection{Systematic Errors}
\begin{enumerate}
	\item .
\end{enumerate}

\subsubsection{Random Errors}
\begin{enumerate}
	\item Uncertanity .. .. 
\end{enumerate}
\subsection{Error Propogation}
To calculate the standard deviation from values we have measured with some least count, we use the formula [2]: 
\begin{equation}
\sigma = \frac{\text{Least Count}}{\sqrt{12}}$$
\end{equation}
and to calculate the standard deviation for values we have applied to the instrument with some least count, we use the formula [2]: 
\begin{equation}
\sigma = \frac{\text{Least Count}}{\sqrt{24}}$$
\end{equation}
\subsection{obs1}



\section{Result and Conclusion}

\vspace{1cm}
\hrule
\nocite{*}
\bibliographystyle{apsrev4-2}
\bibliography{2}
\appendix

\section{Comments}

\end{document}
