\documentclass[%
 sor,
 jor,
 amsmath,amssymb,
 reprint,
]{revtex4-2}

\usepackage{booktabs}
\usepackage{caption}
\usepackage{amsmath, amssymb, amsfonts}
\usepackage{siunitx}
\usepackage{graphicx}
\usepackage{circuitikz}

\begin{document}
\title{PH3244\\Experiment - 2\\BJT characteristics \& applications}
\author{Aumshree P. Shah\\20231059}
\altaffiliation{\color{red}aumshree.pinkalbenshah@students.iiserpune.ac.in}
\date{\today}
\begin{abstract}
\centering
In this experiment we try to see the characteristics of BJT NPN transistor. We also try to construct an oscillator circuit using transistor.
\end{abstract}
\maketitle
\tableofcontents
\vspace*{\fill}
\pagebreak
\section{Theory and Procedure}
\subsection{Theory}


\subsection{Procedure}
Testing the side by diode

\subsubsection{Input Characteristics}
Test the transistor using the DMM as mentioned above  .Connect the transistor in CE mode as per the following circuit-                . current gain                . voltage gain =                             . To test the input characteristics plot input current (IB) versus the input voltage (VBE) for a range  of output voltage VCE  . To test the output characteristics plot the output current (Ic) versus the output voltage (VCE) for  different values of input current IB
\begin{center}
\begin{circuitikz}
\draw (0,0) -- (9,0);
\draw (0,3) to[battery1,l=$V_{BB}$] (0,0);
\draw (6,3) node[npn] (Q1) {};
\draw (Q1.B) -- (4,3);
\draw (4,3) -- (5,3);
\draw (Q1.E) -- (6,0);
\draw (Q1.C) -- (6,4);
\draw (6,4) -- (7,4)
to[R,l=$1.2\,\text{k}\Omega$] (9,4)
      to[battery1,l=$V_{CC}$] (9,0);
\draw (0,3) to[ammeter,l=A, i_=$I_\text{B}$] (2,3)
      to[R,l=$20\,\text{k}\Omega$] (4,3);
      \draw (4,3) to[voltmeter,l=$V_\text{BE}$] (4,0);
\draw (7,4) to[voltmeter,l=$V_\text{CE}$] (7,0);
\end{circuitikz}
\end{center}



\subsubsection{Output Characteristics}
.Test the transistor using the DMM as mentioned above  .Connect the transistor in CE mode as per the following circuit-                . current gain                . voltage gain =                             . To test the input characteristics plot input current (IB) versus the input voltage (VBE) for a range  of output voltage VCE  . To test the output characteristics plot the output current (Ic) versus the output voltage (VCE) for  different values of input current IB
\begin{center}
\begin{circuitikz}
\draw (0,0) -- (10,0);
\draw (0,3) to[battery1,l=$V_{BB}$] (0,0);
\draw (5,3) node[npn] (Q1) {};
\draw (Q1.B) -- (4,3);
\draw (Q1.E) -- (5,0);
\draw (Q1.C) -- (5,4);
\draw (10, 4) to[battery1,l=$V_{CC}$] (10, 0);
\draw (10, 4) to[ammeter, i_=$I_\text{C}$] (8,4); 
\draw (8,4)to[R,l=$1.2\,\text{k}\Omega$] (6,4);
\draw (6,4) -- (5,4);
\draw (0,3) to[ammeter,l=A, i_=$I_\text{B}$] (2,3)
      to[R,l=$20\,\text{k}\Omega$] (4,3);
\draw (6,4) to[voltmeter,l=$V_\text{CE}$] (6,0);
\end{circuitikz}
\end{center}



\subsubsection{Phase Shift Oscillator}
\pagebreak

\section{Observation}

\subsection{Input Characteristics}
\begin{center}
\begin{minipage}{0.31\linewidth}
\centering
\captionof{table}{For $V_\text{CE} =$ 1.00 V}
\begin{tabular}{|c|c|}
\hline
$I_B$ (µA) & $V_{BE}$ (mV) \\
\hline
0 & 3   \\
0 & 280 \\
0 & 441 \\
5 & 632 \\
10 & 663 \\
15 & 672 \\
30 & 675 \\
50 & 677 \\
80 & 679 \\
120 & 683 \\
200 & 689 \\
\hline
\end{tabular}
\end{minipage}
\hfill
\begin{minipage}{0.32\linewidth}
\centering
\begin{tabular}{|c|c|}
\hline
$I_B$ (µA) & $V_{BE}$ (mV)\\
\hline
0 & 3   \\
0 & 198 \\
0 & 358 \\
5 & 583 \\
10 & 634 \\
15 & 636 \\
30 & 639 \\
50 & 644 \\
80 & 649 \\
120 & 656 \\
200 & 667 \\
\hline
\end{tabular}
\captionof{table}{For $V_\text{CE} =$ 4.00 V}
\end{minipage}
\hfill
\hfill
\begin{minipage}{0.34\linewidth}
\centering
\captionof{table}{For $V_\text{CE} =$ 10.00 V}
\begin{tabular}{|c|c|}
\hline
$I_B$ (µA) & $V_{BE}$ (mV)\\
\hline
0 & 3   \\
0 & 335 \\
0 & 553 \\
5 & 636 \\
10 & 661 \\
15 & 672 \\
30 & 696 \\
50 & 701 \\
80 & 704 \\
120 & 706 \\
200 & 710 \\
\hline
\end{tabular}
\end{minipage}
\end{center}




\subsection{Output Characteristics}
\begin{center}
\begin{minipage}{0.31\linewidth}
\centering
\captionof{table}{For $I_\text B = 80$ mA}
\begin{tabular}{|c|c|}
\hline
$V_\text{CE}$ (V) & $I_C$ (mA) \\
\hline
0.004  & 0.00 \\
0.057  & 2.05 \\
0.106  & 7.13 \\
0.177  & 13.84 \\
0.327  & 17.01 \\
0.529  & 18.73 \\
0.823  & 20.32 \\
1.191  & 20.95 \\
1.958  & 21.80 \\
3.205  & 23.09 \\
5.660  & 25.96 \\
9.680  & 29.76 \\
16.760 & 34.46 \\
22.650 & 38.19 \\
\hline
\end{tabular}
\end{minipage}
\hfill
\begin{minipage}{0.3\linewidth}
\centering
\begin{tabular}{|c|c|}
\hline
$V_\text{CE}$ (V) & $I_C$ (mA) \\
\hline
0.005 & 0.00 \\
0.058 & 1.24 \\
0.134 & 7.19 \\
0.208 & 10.25 \\
0.405 & 11.04 \\
0.606 & 12.29 \\
0.844 & 12.38 \\
1.134 & 12.44 \\
2.231 & 12.67 \\
3.262 & 12.94 \\
4.860 & 13.54 \\
7.320 & 14.05 \\
\hline
\end{tabular}
\captionof{table}{For $I_\text B = 50$ mA}
\end{minipage}
\hfill
\begin{minipage}{0.31\linewidth}
\centering
\captionof{table}{For $I_B = 20$ mA}
\begin{tabular}{|c|c|}
\hline
$V_\text{CE}$ (V) & $I_C$ (mA) \\
\hline
0.004 & 0.00 \\
0.012 & 0.02 \\
0.067 & 0.46 \\
0.118 & 1.64 \\
0.243 & 2.88 \\
0.338 & 2.97 \\
0.540 & 2.97 \\
0.798 & 2.97 \\
1.214 & 2.98 \\
1.922 & 2.99 \\
2.829 & 3.01 \\
4.910 & 3.07 \\
9.110 & 3.16 \\
32.530 & 4.59 \\
\hline
\end{tabular}
\end{minipage}
\end{center}

Comments. about the time takenfor stablity, and heating of transistors:w

Also add why taken in decenting order
Then add least count of stuff:w


In calc add why did error i graph 1 and not in 2, also the range clarification
\pagebreak
\section{Calculations and Characteristics}
\subsection{Input Characteristics}
\begin{figure}[h]
    \centering
  %  \includegraphics[width=0.75\linewidth]{inputchar}
    \caption{INPUT CHAR}
    \label{fig:input}
\end{figure}


\subsection{Output Characteristics}
\begin{figure}[h]
    \centering
  %  \includegraphics[width=0.75\linewidth]{outputchar}
    \caption{OUTPUT CHAR}
    \label{fig:output}
\end{figure}


\section{Result and Conclusion}

From the graph of input char we can say it works as a normal diode, turinging on at between range of 0.67 volts 


\appendix


\bibliography{}
\end{document}
